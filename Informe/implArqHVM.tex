\section{Arquitectura del \textit{port} de HVM para las plataformas CIAA} \label{sec:implArqHVM}

Para la implementaci�n de este trabajo se utiliza una arquitectura en capas con responsabilidades e interfaces definidas. El esquema general se presenta en la figura  [\ref{fig:ArquitecturaPort10x9}]. 

\begin{figure}[!htbp]
\begin{center}
\includegraphics*[width=10cm,height=9cm]{figuras/ArquitecturaPort10x9.png}
\par\caption{Arquitectura del port de HVM para las plataformas CIAA.}\label{fig:ArquitecturaPort10x9}
\end{center}
\end{figure}

Las capas preexistentes son \textbf{icecap SDK}, \textbf{HVM}, \textbf{HVM SCJ}, que corresponden a las distintas partes de que conforman HVM; y \textbf{LPCOpen}, la biblioteca provista por NXP para el microcontrolador LPC4337. Las capas desarrolladas se describen a continuaci�n.

\bigskip
\titulo{Vector de interrupciones}

Esta capa se compone de un �nico archivo nombrado \textbf{IsrVector.c} que contiene el vector de interrupciones. Para definir una nueva interrupci�n se debe agregar el nombre de la funci�n correspondiente al \textit{handler} de la misma en el vector, y luego definir dicha funci�n. El �nico \textit{handler} utilizado es el de interrupci�n de \textbf{SysTick}.

\newpage
\titulo{Biblioteca de Perif�ricos C}

La capa Biblioteca de Perif�ricos C, implementa la biblioteca de perif�ricos a bajo nivel, est� dise�ada para abstraer el hardware y presentar una API sencilla para el usuario. Se detalla en el apartado [\ref{sec:implBiblioPerif}]. Utiliza la biblioteca LPCOpen y el vector de interrupci�n.

\bigskip
\titulo{Capas que implementan la HAL de HVM para LPC4337}

La HAL de HVM a bajo nivel se implementa en dos capas. La primera s�lo puede acceder a la Biblioteca de Perif�ricos C, compuesta por los archivos:

\begin{itemize}
\item
\textbf{LPC4337\_peripheral\_natives.c}. Esta capa simplemente implementa las funciones que respetan la firma de los m�todos nativos y llaman a sus respectivas funciones de la Biblioteca de Perif�ricos C.
\item
\textbf{LPC4337\_natives.c}. Contiene las funciones necesarias para el funcionamiento b�sico de HVM, se detalla en la secci�n [\ref{sec:HVMenLPC4337}].
\item
\textbf{LPC4337\_natives\_SCJ.c}. Resuelve las funciones de temporizaci�n para el \textit{Planificador SCJ}. Utiliza el m�dulo \textbf{Tick.c} de la capa Biblioteca de Perif�ricos C. En la secci�n [\ref{sec:HVMscjEnLPC4337}] se comunica su implementaci�n. 
\end{itemize}

\noindent La segunda capa accede directamente al hardware, est� implementada en \textit{assembler} en el archivo \textbf{LPC4337 \_interrupt.s}. Resuelve el cambio de \textit{threads} guardando el contexto y es la base del funcionamiento del m�dulo \textit{Proceso SCJ}. 

\bigskip
\titulo{Biblioteca de Perif�ricos Java}

Corresponde a la implementaci�n de la biblioteca de perif�ricos en Java que finalmente dispondr� el usuario de HVM. Sus m�todos nativos se encuentran implementados en \textbf{LPC4337\_peripheral\_natives.c}.