\section{Implementaci�n de la biblioteca para manejo de perif�ricos} \label{sec:implBiblioPerif}

\section{Biblioteca para manejo de perif�ricos (C)} \label{sec:implBiblioC}

%Implementaci�n de API para manejo de perif�ricos CIAA-NXP.

%sAPI

%Como subproducto, se obtiene adem�s un nuevo m�dulo de Firmware para la CIAA nombrado sAPI (simple API) que permite mediante un conjunto de clases encapsular el manejo de perif�ricos con una interfaz muy sencilla pensada para el programador con perfil inform�tico no experto en Sistemas embebidos. 

M�dulo \textbf{sAPI}:
Est� compuesta por los m�dulos:

\begin{verbatim}
sAPI.h
sAPI_Board.c, sAPI_Board.h
sAPI_DataTypes.h
sAPI_Delay.c, sAPI_Delay.h
sAPI_DigitalIO.c, sAPI_DigitalIO.h
sAPI_IsrVector.c, sAPI_IsrVector.h
sAPI_Tick.c, sAPI_Tick.h
\end{verbatim}

Este actualiza una variable global y luego ejecuta una funci�n cuya direcci�n est� contenida en una variable global para permitir engancharse a la interrupci�n de tick.


\section{Biblioteca para manejo de perif�ricos (Java)} \label{sec:implBiblioJava}

Est� formada por los m�dulos Device.Java, Pin.Java, Peripheral.Java, DigitalIO.Java, AnalogIO.Java, Uart.Java, Delay.Java, Button.Java y Led.Java. 



	\subsection{CIAA-NXP}
         %Mapeo de pines
         
	\subsection{EDU-CIAA}
         %Mapeo de pines
         
	\subsection{Java} %  ---> Como se ve la API desde Java Space
