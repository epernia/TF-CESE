\section{Elecci�n de HVM} \label{sec:eleccionHvm}







\textit{Hardware near Virtual Machine} (HVM)\footnote{HVM desarrollado por Stephan Erbs Korsholm.} es un entorno de ejecuci�n de \textit{Safety Critical} Java (SCJ) nivel 1 y 2, c�digo abierto dise�ado par plataformas embebidas de bajos recursos.

Corre directamente sobre el hardware sin necesidad de un sistema operativo (\textit{bare-metall}). Su dise�o y documentaci�n facilita la portabilidad a nuevas arquitecturas.

Se compone de las siguientes partes:
\begin{itemize}
\item Icecaptools 
\item HVM SDK
\end{itemize}

Para desarrollar aplicaciones sobre HVM se utiliza Icecaptools, el cual es un \textit{plugin} de Eclipse, que convierte a Eclipse en un IDE para la programaci�n en lenguaje Java para HVM sobre sistemas embebidos. 

Icecaptools genera c�digo C a partir de la aplicaci�n Java de usuario para correr sobre la m�quina virtual de Java de HVM, as� como los proprios archivos que implementan a la m�quina virtual.

HVM SDK - El Software Development Kit de HVM que incluye las clases que implementan SCJ.
