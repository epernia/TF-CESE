\section{Elecci�n de la VM de Java, HVM} \label{sec:eleccionHvm}

HVM son las siglas de \textit{Hardware Virtual Machine}. Es una m�quina virtual (en adelante VM) de Java, libre, dise�ada para ejecutarse \textit{bare-metal}\footnote{Significa que se ejecuta directamente sobre el hardware, sin necesidad de un sistema operativo.} sobre microcontroladores y portable a diferentes arquitecturas. Su implementaci�n de referencia fue realizada por Stephan Erbs Korsholm (Icelab, Dinamarca) para su tesis doctoral y puede ejecutarse sobre Posix o AVR. Esta VM cumple con la especificaci�n Safety-Critical Java (SCJ) Level 0, 1 y 2 permitiendo realizar aplicaciones en tiempo real para sistemas cr�ticos. Provee tres formas de acceso al hardware:

\begin{itemize}
\item 
\textbf{Variables Bindeadas.} Es una variable que puedo utilizar en lenguaje Java y tiene correspondencia directa con una variable en otro lenguaje (en este caso particular, lenguaje C).
\item 
\textbf{\textit{Hardware Objects}.} Es una abstracci�n que permite acceder a registros del microcontrolador mapeados en memoria para manipularlos desde el programa en lenguaje Java. De esta forma se puede crear una biblioteca completa dependiente del microcontrolador que maneje un perif�rico a nivel de registros directamente en Java.
\item 
\textbf{M�todos nativos.} Esta alternativa permite utilizar funciones realizadas en otro lenguaje como m�todos en lenguaje Java. De esta manera permite ejecutar c�digo \textit{legacy} dando la posibilidad de utilizar bibliotecas completas realizadas en otro lenguaje. Para conectar un m�todo con una funci�n en lenguaje C, deben respetarse ciertan convenciones de nombres y de pasajes de par�metros en las funciones realizadas para que el compilador de Java puede asociarlas.
\end{itemize}

Se elije m�todos nativos como alternativa para proveer al programa de usuario en lenguaje Java el acceso a los perif�ricos del microcontrolador. Esto es porque ya existen bibliotecas completas de manejo de perif�ricos y adem�s \textit{stacks} y \textit{file systems} entre otras.
