\chapter{RESULTADOS - (8 pag)}\label{cap:resultados}
%�Funca?

%\section{Acerca del IDE obtenido} \label{sec:ideObtenido}

Acerca del IDE obtenido.


%\section{Comparaciones entre Java y C} \label{sec:comparJavaC}

%Comparaciones de tama�o de c�digo, uso de memoria y velocidad de ejecuci�n entre Java y C.

%\section{Aplicaci�n de referencia SCJ} \label{sec:appSCJ}

Aplicaci�n de referencia SCJ.



En las siguientes secciones se exponen distintas aplicaciones que prueban el funcionamiento del desarrollo y sus m�tricas. Estas son:

%%%%%%%%%%%%%%%%%%%%%%%%%%%%%%%%%%%%%%%%%%%%%%%%%%%%%%

\begin{itemize}
\item
Ejemplos de aplicaciones Java utilizando perif�ricos de la CIAA-NXP y EDU-CIAA-NXP mediante la API desarrollada (secci�n [\ref{sec:appPeriph}]).
\item
Un ejemplo de aplicaci�n Java SCJ utilizando el concepto de Proceso SCJ para demostrar el funcionamiento del cambio de contexto (secci�n [\ref{sec:appProcess}]).
\item
Otro ejemplo de aplicaci�n Java SCJ utilizando el concepto de Planificador SCJ (secci�n [\ref{sec:appScheduler}]).
\item 
Una aplicaci�n SCJ completa (secci�n [\ref{sec:appFullSCJ}]).
\end{itemize}

Finalmente se exponen las caracter�sticas del IDE desarrollado (secci�n [\ref{sec:IDEhvm}]).

%%%%%%%%%%%%%%%%%%%%%%%%%%%%%%%%%%%%%%%%%%%%%%%%%%%%%%

\section{Ejemplos de aplicaciones Java utilizando perif�ricos} \label{sec:appPeriph}
\section{Ejemplo de Procesos SCJ} \label{sec:appProcess}
\section{Planificador SCJ} \label{sec:appScheduler}
\section{Ejemplo de aplicaci�n SCJ completa} \label{sec:appFullSCJ}

\section{Acerca del IDE obtenido} \label{sec:IDEhvm}