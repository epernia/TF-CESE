
%{\large \textbf{REFERENCIA BIBLIOGR\'AFICA}}
\chapter*{REFERENCIA BIBLIOGR\'AFICA}

\begin{enumerate}


%%%%%%%%%%%%%%%%%%%%%%  CIAA  %%%%%%%%%%%%%%%%%%%%%%  

% http://proyecto-ciaa.com.ar/devwiki/doku.php?id=start

% http://www.proyecto-ciaa.com.ar/devwiki/doku.php?id=desarrollo:hardware:ciaa_nxp:ciaa_nxp_inicio

%http://proyecto-ciaa.com.ar/devwiki/doku.php?id=desarrollo:edu-ciaa:edu-ciaa-nxp

%%%%%%%%%%%%%%%%%%%%%%  JAVA y SCJ  %%%%%%%%%%%%%%%%  

% PONER COMO BIBLIO DE JAVA - EL LIBRO DE JUAREZ
% PONER COMO BIBLIO DE JAVA - http://www3.uji.es/~belfern/pdidoc/IX26/Documentos/introJava.pdf
% Procesador Java - http://www.atmel.com/images/arm_926ejs_trm.pdf

% RTSJ http://www.rtsj.org/specjavadoc/book_index.html
% RTSJ http://www.dosideas.com/noticias/java/148-java-en-tiempo-real.html

\item 
The Open Group (2013). \emph{Safety-Critical Java Technology Specification JSR-302}. Version 0.94, 25-06-2013. Oracle. On line, �ltima consula 12-10-2015 \url{http://download.oracle.com/otn-pub/jcp/safety_critical-0_94-edr2-spec/scj-EDR2.pdf}.\label{bib:EspecificacionSCJ}


%%%%%%%%%%%%%%%%%%%%%%  HVM  %%%%%%%%%%%%%%%%%%%%%%%  

\item 
Stephan E. Korsholm (2014). \emph{The HVM Reference Manual}. Icelabs, Dinamarca. On line, �ltima consula 05-10-2015 en \url{http://icelab.dk/resources/HVMRef.pdf}.\label{bib:HVMref}

\item 
Hans S�ndergaard, Stephan E. Korsholm y Anders P. Ravn (2012). \emph{Safety-Critical Java for Low-End Embedded Platforms}. Icelabs, Dinamarca. On line, �ltima consula 09-10-2015 en \url{http://icelab.dk/resources/SCJ_JTRES12.pdf}.\label{bib:PaperSCJKorsholm}


%http://www.icelab.dk/ <--- HVM

%http://people.cs.aau.dk/~luckow/hvmtp/ <--- HVM-TP

% BIBLIOGRAFIA
%Luckow, K. S., Thomsen, B., & Korsholm, S. E. (2014). HVMTP: A time predictable and portable java virtual machine for hard real-time embedded systems. In Proceedings of the 12th International Workshop on Java Technologies for Real-Time and Embedded Systems. (pp. 107-116). Association for Computing Machinery. (International Workshop of Java Technologies for Real-Time and Embedded Systems. Proceedings). 10.1145/2661020.2661022
% LINK: http://vbn.aau.dk/en/publications/hvmtp%2864e244eb-9de1-4a1f-962a-9c66ffa2a249%29.html

\item 
Wikipedia, the free encyclopedia. \emph{Dangling pointer}. Wikipedia, the free encyclopedia. On line, �ltima consula 14-10-2015 en \url{https://en.wikipedia.org/wiki/Dangling_pointer}.\label{bib:Dangling_pointer}

\item 
Joseph Yiu (2014). \emph{The Definitive Guide to ARM(R) Cortex(R)-M3 and Cortex(R)-M4 Processors, ed. 3}: Newnes.\label{bib:YiuDefinitiveGuide}



%%%%%%%%%%%%%%%%%%%%%%  OLD  %%%%%%%%%%%%%%%%%%%%%%%  

%\item 
%Barry, R. (2011).\emph{Using the FreeRTOS TM Real Time Kernel NXP LPC17xx Edition ed. 3}: Real Time Engineers Ltd.\label{bib:BarryFreeRTOS}
%
%\item 
%Bergel, A., Cassou, D., Ducasse, S., y Laval, J. (2013). \emph{Deep into Pharo}. Switzerland: Square Bracket Associates. Consultado en 10-10-2013 en \url{http://pharobooks.gforge.inria.fr/PharoByExampleTwo-Eng/latest/PBE2.pdf}.
%
%\item 
%Black, A., Ducasse, S., Nierstrasz, O. y Pollet, D. (2009). \emph{Pharo por Ejemplo}. Switzerland: Square Bracket Associates. Consultado en 10-10-2013 en \url{http://pharobyexample.org/es/PBE1-sp.pdf}.
%
%\item 
%Claus Gittinger Development \& Consulting (1996). \emph{Stream classes}. Consultado en 10-10-2013 en \url{http://live.exept.de/doc/online/english/overview/basicClasses/streams.html}.
%
%\item 
%Ducasse, S. (2008). \emph{[Pharo-project] (Foro) - Collapsing panes}. Consultado en 10-10-2013 en \url{http://lists.gforge.inria.fr/pipermail/pharo-project/2011-January/040520.html}.
%
%\item 
%Gamma, E., Helm, R., Johnson, R. Vlissides, J. (1995). \emph{Patrones de Dise�o: Elementos de software orientado a objetos reutilizable}. Madrid: PEARSON - Addison Wesley.  \label{bib:patronesDiseno}
%
%\item 
%Gemtalk Systems (2011). \emph{Pharo the collaborActibe book}. Consultado en 10-10-2013 en \url{http://pharo.gemtalksystems.com/book/}.
%
%\item 
%Giner, G., Rafael, J. (2008). "\emph{Programaci�n estructurada en C ed. 1}". Pearson Prentice Hall.
%
%\item 
%Gough, B. (2005). An Introduction to GCC. Network Theory Ltd.
%
%\item 
%IEC (2003). \emph{IEC 61131-3 Programmable controllers - Part 3: Programming languages ed2.0}. International Electrotechnical Commission.
%
%\item 
%Juarez, J. (2012). \emph{UNQ - Apuntes de c�tedra: Sistemas Digitales}. Consultado en 10-10-2013 en \url{http://iaci.unq.edu.ar/materias/sistemas_digitales/index.htm}.
%
%\item 
%Kosik, M. (2008). \emph{Pluggable Morphs Demo}. Consultado en 10-10-2013 en \url{http://wiki.squeak.org/squeak/2962}.
%
%\item 
%Lewis, D. (2008). \emph{OSProcess}. Consultado en 10-10-2013 en \url{http://wiki.squeak.org/squeak/708}.
%
%\item 
%LSE (2012). \emph{UBA - Apuntes de c�tedra: Sistemas embebidos}. Consultado en 10-10-2013 en \url{http://laboratorios.fi.uba.ar/lse/}.
%
%\item 
%Moreno Gomez, J. (2009). \emph{What is the difference between Sinking and Sourcing Input Configuration - PLC?}. Consultado en 10-10-2013 en \url{http://reliability-maintenance.blogspot.com.ar/2009/07/what-is-difference-between-sinking-and.html}.
%
%\item 
%National Instruments (2011). \emph{Digital I/O Sinking and Sourcing}. Consultado en 10-10-2013 en \url{http://www.ni.com/white-paper/3291/en}.
%
%\item 
%Radioaficionados (2010). \emph{Protecci�n contra inversiones de polaridad}. Consultado en 10-10-2013 en \url{http://www.radioelectronica.es/radioaficionados/19-inversion-polaridad}.
%
%\item 
%Ritchie, D. (1993). \emph{The Development of the C Language}. Consultado en 10-10-2013 en \url{http://cm.bell-labs.com/cm/cs/who/dmr/chist.html}.
%
%\item 
%Siemens Industry Online Support (2013). \emph{�Qu� significan los t�rminos "sumidero" (alem�n: "P-schaltend") y "fuente" (alem�n: "M-schaltend") en los m�dulos digitales de SIMATIC?}. Consultado en 10-10-2013 en \url{http://support.automation.siemens.com/WW/llisapi.dll?func=cslib.csinfo\&objId=42616517\&load=treecontent&lang=es\&siteid=cseus\&aktprim=0\&objaction=csview\&extranet=standard&viewreg=WW}.
%
%\item 
%Stallman, R., Pesch, R., Shebs, S., et al. (2011). \emph{Debugging with GDB}. Free Software Foundation. 
%
%\item 
%Stallman, R. (2001). \emph{Using and Porting the GNU Compiler Collection}. Free Software Foundation. Consultado en 10-10-2013 en \url{http://gcc.gnu.org/onlinedocs/gcc-2.95.3/gcc.html}.
%
%\item 
%Szirty (2013). \emph{PLC programoz�s sokf�lek�ppen (La programaci�n de PLC de muchas maneras)}. Consultado en 10/10/2013 en \url{http://szirty.taviroda.com/lang/index.html}. \label{bib:Szirty}
%
%\item 
%Wikipedia (2013). \emph{Analizador sint�ctico}. Consultado en 10-10-2013 en \url{http://es.wikipedia.org/wiki/Analizador_sint\%C3\%A1ctico}.
%
%\item 
%Wikipedia (2013). \emph{Drop-down list}. Consultado en 10-10-2013 en \url{http://en.wikipedia.org/wiki/Drop-down\_list}.
%
%\item 
%Wikipedia (2013). \emph{Framework}. Consultado en 10-10-2013 en \url{http://es.wikipedia.org/w/index.php?title=Framework\&section=10}.
%
%\item 
%Wikipedia (2013). \emph{Programaci�n orientada a objetos}. Consultado en 10-10-2013 en \url{http://es.wikipedia.org/wiki/Programacion\_orientada\_a\_objetos}.
%
%\item 
%Wikipedia (2013). \emph{Smalltalk}. Consultado en 10-10-2013 en \url{http://es.wikipedia.org/wiki/Smalltalk}.
%
%\item 
%Wikipedia (2013). \emph{Tuber�a (inform�tica)}. Consultado en 10-10-2013 en \url{http://es.wikipedia.org/w/index.php?title=Tuber\%C3\%ADa_\%28inform\%C3\%A1tica\%29}.

\end{enumerate}

\thispagestyle{empty}