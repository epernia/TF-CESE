\section{IDE Icecaptools} \label{sec:icecaptools}

%http://www.icelab.dk/ <--- HVM

%http://people.cs.aau.dk/~luckow/hvmtp/ <--- HVM-TP

% BIBLIOGRAFIA
%Luckow, K. S., Thomsen, B., & Korsholm, S. E. (2014). HVMTP: A time predictable and portable java virtual machine for hard real-time embedded systems. In Proceedings of the 12th International Workshop on Java Technologies for Real-Time and Embedded Systems. (pp. 107-116). Association for Computing Machinery. (International Workshop of Java Technologies for Real-Time and Embedded Systems. Proceedings). 10.1145/2661020.2661022
% LINK: http://vbn.aau.dk/en/publications/hvmtp%2864e244eb-9de1-4a1f-962a-9c66ffa2a249%29.html



HVM provee tres formas de acceso al hardware:

\begin{itemize}
\item 
\textbf{Variables Bindeadas.} Es una variable que puedo utilizar en lenguaje Java y tiene correspondencia directa con una variable en otro lenguaje (en este caso particular, lenguaje C).
\item 
\textbf{\textit{Hardware Objects}.} Es una abstracci�n que permite acceder a registros del microcontrolador mapeados en memoria para manipularlos desde el programa en lenguaje Java. De esta forma se puede crear una biblioteca completa dependiente del microcontrolador que maneje un perif�rico a nivel de registros directamente en Java.
\item 
\textbf{M�todos nativos.} Esta alternativa permite utilizar funciones realizadas en otro lenguaje como m�todos en lenguaje Java. De esta manera permite ejecutar c�digo \textit{legacy} dando la posibilidad de utilizar bibliotecas completas realizadas en otro lenguaje. Para conectar un m�todo con una funci�n en lenguaje C, deben respetarse ciertan convenciones de nombres y de pasajes de par�metros en las funciones realizadas para que el compilador de Java puede asociarlas.
\end{itemize}

Se elije m�todos nativos como alternativa para proveer al programa de usuario en lenguaje Java el acceso a los perif�ricos del microcontrolador. Esto es porque ya existen bibliotecas completas de manejo de perif�ricos y adem�s \textit{stacks} y \textit{file systems} entre otras.
