\section{IDE Icecaptools} \label{sec:icecaptools}

%http://www.icelab.dk/ <--- HVM

%http://people.cs.aau.dk/~luckow/hvmtp/ <--- HVM-TP

% BIBLIOGRAFIA
%Luckow, K. S., Thomsen, B., & Korsholm, S. E. (2014). HVMTP: A time predictable and portable java virtual machine for hard real-time embedded systems. In Proceedings of the 12th International Workshop on Java Technologies for Real-Time and Embedded Systems. (pp. 107-116). Association for Computing Machinery. (International Workshop of Java Technologies for Real-Time and Embedded Systems. Proceedings). 10.1145/2661020.2661022
% LINK: http://vbn.aau.dk/en/publications/hvmtp%2864e244eb-9de1-4a1f-962a-9c66ffa2a249%29.html


Icecaptools se distribuye como un plugin de Eclipse realizado por Stephan Erbs Korsholm, que convierte al Eclipse en un IDE para la programaci�n en lenguaje Java y permite compilar el programa de usuario para HVM. Funciona realizando una traducci�n de un programa de usuario escrito en lenguaje Java, a un programa en lenguaje C que incluye el c�digo de dicho programa y el c�digo C generado de HVM. De esta manera, logra portabilidad entre diferentes microcontroladores y permite integraci�n con programas escritos previamente en lenguaje C, como por ejemplo, el firmware de la CIAA. Requiere un toolchain de lenguaje C, para el microcontrolador objetivo, que permita compilar el c�digo, generando un binario ejecutable, y su posterior descarga a dicho dispositivo. En la figura [\ref{fig:Icecaptools}] se incluye un esquema gr�fico de su funcionamiento.

\begin{figure}[!htbp]
\begin{center}
\includegraphics*[width=12cm,height=12cm]{figuras/Icecaptools.png}
\par\caption{Esquema de funcionamiento de Icecaptools.}\label{fig:Icecaptools}
\end{center}
\end{figure}