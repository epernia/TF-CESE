\section{Port de HVM SCJ al microcontrolador NXPLPC4337} \label{sec:HVMscjEnLPC4337}

En esta etapa final del \textit{port} se definen las funciones nativas de la HAL de HVM, necesarias para poder realizar programas SCJ.

%--------------------------------------------------
\bigskip
\titulo{Archivo \textbf{LPC4337\_natives\_SCJ.c}}

Estas funciones utilizan las provistas en el archivo \textbf{sapi\_Tick.c} de la biblioteca de C.

\begin{itemize}
\item
\textbf{void start\_system\_tick(void)}, \textbf{void stop\_system\_tick(void)}.  Estas funciones inician utilizan la funci�n \textbf{void tickConfig(uint64_t tickRateHz)}, la misma recibe un par�metro que utiliza para configurar el perif�rico \textbf{SysTick}, si el mismo es 0, frena la interrupci�n de \textit{SysTick}. 
\item
\textbf{int16 n\_vm\_RealtimeClock\_awaitNextTick(int32 *sp)}. Esta funci�n queda en un bucle hasta que cambie el valor de la variable global \textbf{systemTick}. 
\item
\textbf{int16 n\_vm\_RealtimeClock\_getNativeResolution(int32 *sp)}. La funci�n \textbf{void tickConfig(uint32_t tickRateHz)} almacena el valor \textbf{tickRateHz} recibido como par�metro en un variable global, este valor es utilizado para calcular el tiempo entre dos \textit{ticks}.
\item
\textbf{int16 n\_vm\_RealtimeClock\_getNativeTime(int32 *sp)}. Devuelve el tiempo actual del reloj de tiempo real como un objeto AbsoluteTime con mili segundos y nano segundos.
\end{itemize}


%--------------------------------------------------
\bigskip
\titulo{Archivo \textbf{LPC4337\_interrupt.s}}


\noindent Luego, en el archivo \textbf{XX\_interrupt.s} deben realizarse tres funciones en \textit{assembler} necesarias para implementar los \textbf{procesos SCJ} y el cambio de \textit{threads} (cambio de contexto).

\medskip

La funci�n \textbf{\_yield}. Debe guardar todos los registros en la pila, guardar el puntero a pila en la variable global \textbf{stackPointer} (declarada en natives\_allOS.c) y llama a la funci�n \textbf{transfer} (definida tambi�n en natives\_allOS.c). Cuando termina la ejecuci�n de \textbf{transfer} debe guardar el valor de la variable global \textbf{stackPointer} al puntero a pila, restaurar todos los registros (en orden inverso) y retornar.

\medskip
   
Cuando se llama a la funci�n \textbf{pointer* get\_stack\_pointer(void)} debe retornar el valor del puntero a pila. Los pasos para llevarlo a cabo son:

\begin{enumerate}
\item
Mover el valor del puntero a pila al registro utilizado por el compilador para el valor de retorno de funciones.
\item
El valor actual del puntero a pila es el valor del \textit{frame} actual. Se debe ajustar el valor de retorno ya que se requiere el valor del \textit{frame} que llam� a esta funci�n.
\item
Devolver este �ltimo valor de retorno.
\end{enumerate}

La funci�n \textbf{set\_stack\_pointer(void)} Debe establecer el valor de la variable global \textbf{stackPointer} en el puntero a pila y retornar a la funci�n invocante. Concretamente:

\begin{enumerate}
\item
Mover el valor de retorno de la pila a alg�n registro.
\item
Mover el valor de la variable global \textbf{stackPointer} al registro puntero a pila.
\item
Mover el valor de retorno guardado en 1 a la pila.
\item
Retornar.
\end{enumerate}

Por lo general (en la mayor�a de las arquitecturas) al llamar a una funci�n se inserta en la pila la direcci�n de retorno. Esta es la direcci�n donde debe continuar al ejecuci�n cuando termine de ejecutar la funci�n que ha llamado.
Luego cuando se retorna de la funci�n, se saca de la pila la direcci�n de retorno y se realiza un salto a dicha direcci�n.
Esto provoca que la ejecuci�n contin�e en la direcci�n correcta al terminar de ejecutar la funci�n. 
En la funci�n \textbf{set\_stack\_pointer} se utiliza una nueva pila, sin embargo, se necesita de todas formas retornar la direcci�n de donde se ha llamado a la funci�n \textbf{set\_stack\_pointer}. 
Como se establece un nuevo puntero a pila, la direcci�n de retorno correcta no se encuentra en esta nueva pila. Por esto, es necesario que en el paso 1 se mueva la direcci�n de retorno a la nueva pila para poder retornar correctamente a donde se ejecut� \textbf{set\_stack\_pointer}.

\medskip

En la implementaci�n este archivo se nombra \textbf{LPC4337\_interrupt.s}