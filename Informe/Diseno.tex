\chapter{DISE�O - (8 pag)} \label{cap:diseno}

%%%%%%%%%%%%%%%%%%%%%%%%%%%%%%%%%%%%%%%%%%%%%%%%%%%%%

\noindent En particular, la implementaci�n consiste en:

\begin{itemize}
\item 
La realizaci�n del \textit{port} de la m�quina virtual de HVM para que corra sobre el microcontrolador NXP LPC4337, que contienen las plataformas CIAA-NXP y EDU-CIAA-NXP, permitiendo la programaci�n de aplicaciones Java.
\item 
Un dise�o e implementaci�n de una API\footnote{\textit{Application Programming Interface}, es decir, una interfaz de programaci�n de aplicaciones.} sencilla para permitir controlar el Hardware desde una aplicaci�n Java, que funciona adem�s, como HAL\footnote{\textit{Hardware Abstraction Layer}, significa: capa de abstracci�n de hardware.}.
\item 
El \textit{port} de la capa SCJ de la m�quina virtual de HVM, para permitir desarrollar aplicaciones Java SCJ.
\item 
La integraci�n del \textit{port} para la CIAA al IDE de HVM, para completar un IDE de Java SCJ sobre la CIAA.
\end{itemize}

%%%%%%%%%%%%%%%%%%%%%%%%%%%%%%%%%%%%%%%%%%%%%%%%%%%%%



\section{IDE Icecaptools} \label{sec:icecaptools}

%http://www.icelab.dk/

Icecaptools se distribuye como un plugin de Eclipse realizado por Stephan Erbs Korsholm, que convierte al Eclipse en un IDE para la programaci�n en lenguaje Java y permite compilar el programa de usuario para HVM. Funciona realizando una traducci�n de un programa de usuario escrito en lenguaje Java, a un programa en lenguaje C que incluye el c�digo de dicho programa y el c�digo C generado de HVM. De esta manera, logra portabilidad entre diferentes microcontroladores y permite integraci�n con programas escritos previamente en lenguaje C, como por ejemplo, el firmware de la CIAA. Requiere un toolchain de lenguaje C, para el microcontrolador objetivo, que permita compilar el c�digo, generando un binario ejecutable, y su posterior descarga a dicho dispositivo. En la figura 1 se incluye un esquema gr�fico de su funcionamiento.
\section{Porting de HVM e icecaptools a una nueva arquitecura} \label{sec:portingHvm}

Para portar el Firmware HVM a una nueva arquitectura deben realizarse los siguientes pasos:

\begin{itemize}
\item 
Conseguir un ambiente de desarrollo en lenguaje C para la arquitectura.
\item 
Estandarizar los tama�os de tipos de datos.
\item 
Implementar una HAL para la arquitectura.
\item 
Implementar el acceso al hardware.
\end{itemize}

Para facilitar la inclusi�n de nuevas arquitecturas a icecaptools y automatizar tareas habituales, que en caso contrario deber�a realizar el usuario para cada nueva aplicaci�n, se propone:

\begin{itemize}
\item 
Adaptar Icecaptools para la arquitectura ARM mediante su extensi�n a trav�s de un plug-in que llamaremos HvmForArmCortexM.
\item 
La programaci�n de un segundo plug-in m�s espec�fico llamado HVM4CIAA.
\end{itemize}

Finalmente, en la figura 2 se gr�fica la soluci�n propuesta para lograr utilizar icecaptools y HVM con la CIAA.

FIGURAAAAAAAAAAA

\input{HalPerifericos}
\section{Dise�o del IDE} \label{sec:disenoIde}

Dise�o del IDE.

