\chapter{DISE�O - (8 pag)} \label{cap:diseno}

%%%%%%%%%%%%%%%%%%%%%%%%%%%%%%%%%%%%%%%%%%%%%%%%%%%%%

\noindent En particular, la implementaci�n consiste en:

\begin{itemize}
\item 
La realizaci�n del \textit{port} de la m�quina virtual de HVM para que corra sobre el microcontrolador NXP LPC4337, que contienen las plataformas CIAA-NXP y EDU-CIAA-NXP, permitiendo la programaci�n de aplicaciones Java.
\item 
Un dise�o e implementaci�n de una API\footnote{\textit{Application Programming Interface}, es decir, una interfaz de programaci�n de aplicaciones.} sencilla para permitir controlar el Hardware desde una aplicaci�n Java, que funciona adem�s, como HAL\footnote{\textit{Hardware Abstraction Layer}, significa: capa de abstracci�n de hardware.}.
\item 
El \textit{port} de la capa SCJ de la m�quina virtual de HVM, para permitir desarrollar aplicaciones Java SCJ.
\item 
La integraci�n del \textit{port} para la CIAA al IDE de HVM, para completar un IDE de Java SCJ sobre la CIAA.
\end{itemize}

%%%%%%%%%%%%%%%%%%%%%%%%%%%%%%%%%%%%%%%%%%%%%%%%%%%%%



\section{IDE Icecaptools} \label{sec:icecaptools}

%http://www.icelab.dk/ <--- HVM

%http://people.cs.aau.dk/~luckow/hvmtp/ <--- HVM-TP

% BIBLIOGRAFIA
%Luckow, K. S., Thomsen, B., & Korsholm, S. E. (2014). HVMTP: A time predictable and portable java virtual machine for hard real-time embedded systems. In Proceedings of the 12th International Workshop on Java Technologies for Real-Time and Embedded Systems. (pp. 107-116). Association for Computing Machinery. (International Workshop of Java Technologies for Real-Time and Embedded Systems. Proceedings). 10.1145/2661020.2661022
% LINK: http://vbn.aau.dk/en/publications/hvmtp%2864e244eb-9de1-4a1f-962a-9c66ffa2a249%29.html



HVM provee tres formas de acceso al hardware:

\begin{itemize}
\item 
\textbf{Variables Bindeadas.} Es una variable que puedo utilizar en lenguaje Java y tiene correspondencia directa con una variable en otro lenguaje (en este caso particular, lenguaje C).
\item 
\textbf{\textit{Hardware Objects}.} Es una abstracci�n que permite acceder a registros del microcontrolador mapeados en memoria para manipularlos desde el programa en lenguaje Java. De esta forma se puede crear una biblioteca completa dependiente del microcontrolador que maneje un perif�rico a nivel de registros directamente en Java.
\item 
\textbf{M�todos nativos.} Esta alternativa permite utilizar funciones realizadas en otro lenguaje como m�todos en lenguaje Java. De esta manera permite ejecutar c�digo \textit{legacy} dando la posibilidad de utilizar bibliotecas completas realizadas en otro lenguaje. Para conectar un m�todo con una funci�n en lenguaje C, deben respetarse ciertan convenciones de nombres y de pasajes de par�metros en las funciones realizadas para que el compilador de Java puede asociarlas.
\end{itemize}

Se elije m�todos nativos como alternativa para proveer al programa de usuario en lenguaje Java el acceso a los perif�ricos del microcontrolador. Esto es porque ya existen bibliotecas completas de manejo de perif�ricos y adem�s \textit{stacks} y \textit{file systems} entre otras.

\section{Port de HVM a una nueva arquitectura} \label{sec:disenoPortHvm} 


Para portar el Firmware HVM a una nueva arquitectura deben realizarse los siguientes pasos:

\begin{itemize}
\item 
Conseguir un ambiente de desarrollo en lenguaje C para la arquitectura.
\item 
Estandarizar los tama�os de tipos de datos.
\item 
Implementar una API para la arquitectura.
\item 
Implementar el acceso al hardware.
\end{itemize}

Para facilitar la inclusi�n de nuevas arquitecturas a icecaptools y automatizar tareas habituales, que en caso contrario deber�a realizar el usuario para cada nueva aplicaci�n, se propone la programaci�n de un plugin llamado HVM4CIAA.


Finalmente, en la figura [] se gr�fica la soluci�n propuesta para lograr utilizar icecaptools y HVM con la CIAA.

\medskip

FIGURAAAAAAAAAAA



	\subsection{Port de HVM para ejecutar Java}
	\subsection{Port de HVM para ejecutar Java SCJ}


\section{Dise�o de HAL para manejo de perif�ricos} \label{sec:halPerifericos}

Dise�o de una HAL para manejo de perif�ricos con una API sencilla.


\section{Dise�o de IDE para aplicaciones Java sobre HVM} \label{sec:disenoIDEhvm}