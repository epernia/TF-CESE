\section{HVM} \label{sec:disenoHVMintro}

El prop�sito de HVM es permitir programar en lenguaje Java dispositivos embebidos con pocos recursos. El c�digo ANSI C generado puede compilarse utilizando un \textit{cross compiler} para el dispositivo en particular para generar el ejecutable.

\medskip

Los recursos m�nimos necesarios en un microcontrolador son 10kB de ROM y 512 bytes de RAM. Sin embargo, para ejecutar programas de tama�o razonables, se necesitan 32 kB de ROM y 2kB de RAM.

\bigskip

HVM funciona realizando una traducci�n de un programa de usuario escrito en lenguaje Java, a un programa en lenguaje C que incluye el c�digo de dicho programa y el c�digo C generado de HVM. De esta manera, logra portabilidad entre diferentes microcontroladores y permite integraci�n con programas escritos previamente en lenguaje C, como por ejemplo, el firmware de la CIAA. Requiere un toolchain de lenguaje C, para el microcontrolador objetivo, que permita compilar el c�digo, generando un binario ejecutable, y su posterior descarga a dicho dispositivo. El proceso completo se ilustra en la figura [\ref{fig:Icecaptools}]. 

\begin{figure}[!htbp]
\begin{center}
\includegraphics*[width=12cm,height=12cm]{figuras/Icecaptools.png}
\par\caption{Esquema de funcionamiento del IDE para trabajar con HVM.}\label{fig:Icecaptools}
\end{center}
\end{figure}

%%%%%%%%%%%%%%%%%%%%%%%%%%%%%%%%%%%%%%

EXPLICAR LA FIGURAAAAAAAA

%%%%%%%%%%%%%%%%%%%%%%%%%%%%%%%%%%%%%%




\subsection{Obtenci�n de un IDE para desarrollar programas Java SCJ sobre HVM}

En la secci�n [\ref{sec:VMsJava}] se adelant� que HVM se distribuye como un \textit{plugin} de Eclipse para convertirlo en un IDE para desarrollar programas Java SCJ sobre HVM. Para su utilizaci�n se debe descargar:

\begin{itemize}
\item
IDE Eclipse. En particular la distribuci�n \textbf{Eclipse Automotive}, recomendada pues integra el desarrollo de aplicaciones Java y C.
\item
El \textit{plugin} de Eclipse de HVM \textbf{Icecaptools}. 
\item
\textbf{HVM SDK}.
\end{itemize}

\textbf{Eclipse Automotive} se descarga en \url{http://ECLIPSEEEEE}. Los otros dos se distribuyen como archivos \textit{.jar} y pueden descargarse de \url{http://icelab.dk/download.html} sus respectivos nombres son \textbf{icecaptools\_x.y.z.jar} e \textbf{icecapSDK.jar}.

\medskip

Una vez que se descarga y descomprime Eclipse se debe instalar sobre el mismo el \textit{plugin} \textbf{icecaptools}, que es el encargado de compilar el programa Java para utilizarse sobre HVM. Para realizar programas SCJ debe incluirse \textbf{icecapSDK} como biblioteca (Jar externa) al proyecto de aplicaci�n Java.

\medskip

Finalmente se debe conseguir un compilador de lenguaje C para la arquitectura. Este compilador puede ser un  programa independiente o integrarse a Eclipse. De esta manera se completan las herramientas requerids para trabajar con HVM.


%%%%%%%%%%%%%%%


TRADUCTOR

HVM es un compilador de Java a C con un int�rprete embebido. Esto significa que Permite traducir un programa Java a un programa en C. De esta manera, la entrada al proceso de traducci�n es un conjunto de archivos fuente en lenguaje Java y la salida un conjunto de archivos fuente en lenguaje C.


\section{IDE Icecaptools} \label{sec:icecaptools}

%http://www.icelab.dk/

Icecaptools se distribuye como un plugin de Eclipse realizado por Stephan Erbs Korsholm, que convierte al Eclipse en un IDE para la programaci�n en lenguaje Java y permite compilar el programa de usuario para HVM. Funciona realizando una traducci�n de un programa de usuario escrito en lenguaje Java, a un programa en lenguaje C que incluye el c�digo de dicho programa y el c�digo C generado de HVM. De esta manera, logra portabilidad entre diferentes microcontroladores y permite integraci�n con programas escritos previamente en lenguaje C, como por ejemplo, el firmware de la CIAA. Requiere un toolchain de lenguaje C, para el microcontrolador objetivo, que permita compilar el c�digo, generando un binario ejecutable, y su posterior descarga a dicho dispositivo. En la figura 1 se incluye un esquema gr�fico de su funcionamiento.

	\subsection{Componentes}
	
	\subsection{Modo de utilizaci�n}

	\subsection{Caracter�sticas}
	

