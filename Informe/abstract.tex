%\chapter*{ABSTRACT}
\newpage
\text{{\color{white}.}}\linebreak \linebreak \linebreak \linebreak \linebreak
\noindent {\huge\textbf{ABSTRACT}} \linebreak \linebreak \linebreak 

The purpose of this final work is the incorporation of new technologies in industrial environments by developing innovative embedded systems architectures. In particular, allow to create Real-Time applications for industrial environments using an object-oriented programming language (hereinafter OOP), on the Argentine Open Industrial Computer (CIAA). It is also expected to bring computer programmers to embedded systems programming discipline, enabling them to apply advanced programming techniques.

\medskip

In this work it was chosen Java as OOP language, and HVM \footnote{acronym for \textit{Hardware near Virtual Machine}, development of Stephan Erbs Korsholm, Denmark.}, a \textit{Safety-Critical Java}\footnote{The Java Safety-Critical specification is an extension to the RTSJ specification, a Java specification for real-time applications} (SCJ)[\ref{bib:HVMref}] execution environment, open source, designed for low resorurces embedded platforms. This work consists in the implementation and validation of a Firmware and Software environment, based on HVM, for programming the CIAA-NXP and EDU-CIAA-NXP platforms in Java SCJ language.

\medskip
\noindent Essentially, the implementation is:

\begin{itemize}
\item 
The port of HVM to run on NXP LPC4337 microcontroller, which contain the CIAA-NXP and EDU-CIAA-NXP platforms, to allow Java applications programming.
\item 
Design and implementation of a library with a simple API\footnote{Application Programming Interface.} to allow hardware use directly in Java space, the library also works as HAL\footnote{Hardware Abstraction Layer.}.
\item 
The port of HVM SCJ layer to allow Java SCJ applications development.
\item 
The manual integration of CIAA port in HVM's IDE by description of necessary steps to work with HVM.
\end{itemize}

\noindent To validate this development it presents:

\begin{itemize}
\item
An example of a Java application that use peripherals of the EDU-CIAA-NXP platform.
\item
Several examples of Java SCJ applications.
\end{itemize}

%% Faltar�an poner los resultados , que creo que ser�a la validaci�n

In conclusion, this final work provides a development environment for Java SCJ applications on the CIAA-NXP and  EDU-CIAA-NXP platforms, it's a free software, and cover all the needs by providing object-oriented programming, and real-time capabilities for embedded systems on industrial environments.

%\end{abstract}