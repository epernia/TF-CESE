%\chapter*{ABSTRACT}
\newpage
\text{{\color{white}.}}\linebreak \linebreak \linebreak \linebreak \linebreak
\noindent {\huge\textbf{ABSTRACT}} \linebreak \linebreak \linebreak 

The purpose of this Final Work is the incorporation of new technologies in industrial environments by developing innovative architectures for embedded systems. In particular, creating industrial Real-Time applications using an Object Oriented Programming language (hereinafter OOP), for execution on the \textit{Computadora Industrial Abierta Argentina} (CIAA) embedded computer. It is also expected to bring traditional computer programmers into embedded systems programming arena, thus enabling to apply advanced programming techniques into them.

\medskip

To carry this out Java was chosen as target OOP language, along with HVM\footnote{Acronym for \textit{Hardware near Virtual Machine}, development of Stephan Erbs Korsholm, Denmark.}, which is an open source \textit{Safety-Critical Java}\footnote{The Java Safety-Critical specification is an extension to the RTSJ specification, a Java specification for real-time applications} (SCJ) execution environment  [\ref{bib:HVMref}] designed for low resource embedded platforms. This work thus consists in the implementation and validation of a Firmware and Software environment based on HVM, to enable programming using SCJ Java language into CIAA-NXP and EDU-CIAA-NXP platforms.

\medskip
\noindent Basically, the implementation consists of:

\begin{itemize}
\item 
The port of HVM to run on NXP LPC4337 microcontroller, which contain the CIAA-NXP and EDU-CIAA-NXP platforms, to allow Java applications programming.
\item 
Design and implementation of a library with a simple API\footnote{Application Programming Interface.} to allow hardware use directly in Java space, the library also works as HAL\footnote{Hardware Abstraction Layer.}.
\item 
The port of HVM SCJ layer to allow Java SCJ applications development.
\item 
The manual integration of CIAA port in HVM's IDE by description of necessary steps to work with HVM.
\end{itemize}

\noindent In order to validate this development, the following examples are presented:

\begin{itemize}
\item
An example of a Java application that use peripherals of the EDU-CIAA-NXP platform.
\item
Several examples of Java SCJ applications.
\end{itemize}

In conclusion, the main contribution of this Final Work is the implementation of a development environment for developing SCJ Java applications onto the CIAA-NXP and EDU-CIAA-NXP platforms. It is presented under open source licensing scheme, and covers the goals of both providing object-oriented programming and real-time capabilities for industrial embedded systems.

%\end{abstract}