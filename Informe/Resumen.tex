\chapter*{RESUMEN}

%En una p�gina explicar la idea del proyecto:
%\\
%\\Qu� se hizo 
%\\Por qu�
%\\C�mo
%\\Resultado
%\\
%Es un dise�o y prueba de concepto, no un producto. Actualmente los entornos gr�ficos de programaci�n de PLC son software propietarios provistos pro cada fabricante de PLC. 

\thispagestyle{empty}

%\begin{abstract} %abstract Funciona solo en article

El prop�sito de este Trabajo Final es la incorporaci�n de nuevas tecnolog�as en ambientes industriales mediante el desarrollo de arquitecturas de sistemas embebidos novedosas. En particular, permitir crear aplicaciones Real-Time para entornos industriales, utilizando un lenguaje de programaci�n orientado a objetos (POO), sobre la Computadora Industrial Abierta Argentina (CIAA). Adem�s, se espera acercar a programadores inform�ticos a la disciplina de programaci�n de sistemas embebidos, permiti�ndoles aplicar t�cnicas avanzadas de programaci�n. 

\medskip

Para llevarlo a cabo se ha escogido Java como lenguaje POO y Icecaptools. Icecaptools es un plug-in de Eclipse realizado por Stephan Erbs Korsholm, que convierte al Eclipse en un IDE para la programaci�n en lenguaje Java y permite compilar el programa de usuario para HVM. HVM es una m�quina virtual de Java, libre, escrita en lenguaje C, dise�ada para correr directamente sobre el hardware. Adem�s, HVM cumple con la especificaci�n Safety-Critical Java (SCJ), permitiendo realizar aplicaciones cr�ticas Real-Time.

\medskip

Se incluye en este trabajo la implementaci�n y validaci�n de un ambiente de Firmware y Software, basado en Icecaptools y HVM, para programar la Computadora Industrial Abierta Argentina (CIAA) en lenguaje Java, para aplicaciones industriales en tiempo real. Este consiste en:

\begin{itemize}
\item 
La realizaci�n del \textit{porting} de HVM para que corra sobre el microcontrolador NXP LPC4337 que contienen las plataformas CIAA-NXP y EDU-CIAA-NXP.
\item 
Un dise�o e implementaci�n de una HAl con API sencilla para permitir controlar el Hardware desde una aplicaci�n Java.
\item 
La realizaci�n del \textit{porting} de la capa SCJ de HVM para que corra sobre el microcontrolador NXP LPC4337 permitiendo aplicaciones SCJ.
\item 
El desarrollo de la integraci�n en Icecaptools del \textit{porting} de HVM para completar el IDE de Java SCJ sobre la CIAA.
\end{itemize}

Para validar el IDE desarrollado se presentan:

\begin{itemize}
\item
Ejemplos de aplicaciones Java utilizando perif�ricos de la CIAA y EDU-CIAA.
\item 
Una aplicaci�n Real-Time (SCJ) de referencia para demostrar el funcionamiento real-time del sistema.
\end{itemize}

%\end{abstract}