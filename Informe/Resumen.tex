\chapter*{RESUMEN}

%En una p�gina explicar la idea del proyecto:
%\\
%\\Qu� se hizo 
%\\Por qu�
%\\C�mo
%\\Resultado
%\\
%Es un dise�o y prueba de concepto, no un producto. Actualmente los entornos gr�ficos de programaci�n de PLC son software propietarios provistos pro cada fabricante de PLC. 

\thispagestyle{empty}

%\begin{abstract} %abstract Funciona solo en article

El prop�sito de este Trabajo Final es la incorporaci�n de nuevas tecnolog�as en ambientes industriales mediante el desarrollo de arquitecturas de sistemas embebidos novedosas. En particular, permitir crear aplicaciones \textit{Real-Time} para entornos industriales, utilizando un lenguaje de programaci�n orientado a objetos (en adelante POO), sobre la Computadora Industrial Abierta Argentina (CIAA). Adem�s, se espera acercar a programadores inform�ticos a la disciplina de programaci�n de sistemas embebidos, permiti�ndoles aplicar t�cnicas avanzadas de programaci�n.

\medskip

Para llevarlo a cabo se ha escogido Java como lenguaje POO, y HVM\footnote{HVM desarrollado por Stephan Erbs Korsholm.}, que es un entorno de ejecuci�n de \textit{Safety Critical Java}\footnote{La especificaci�n \textit{Safety Critica Java} es una extensi�n a la especificaci�n RTSJ, una especificaci�n de java para aplicaciones en tiempo real.} (SCJ)[\ref{bib:HVMref}], de c�digo abierto, dise�ado para plataformas embebidas de bajos recursos. Este trabajo consiste entonces, en la implementaci�n y validaci�n de un ambiente de Firmware y Software, basado en HVM, para programar las plataformas CIAA-NXP y EDU-CIAA-NXP en lenguaje Java SCJ. 

\medskip
\noindent Fundamentalmente, la implementaci�n consiste en:

\begin{itemize}
\item 
La realizaci�n del \textit{port} de la m�quina virtual de HVM para que corra sobre el microcontrolador NXP LPC4337, que contienen las plataformas CIAA-NXP y EDU-CIAA-NXP, permitiendo la programaci�n de aplicaciones Java.
\item 
Un dise�o e implementaci�n de una biblioteca con API\footnote{\textit{Application Programming Interface}, es decir, una interfaz de programaci�n de aplicaciones.} sencilla para permitir controlar el Hardware desde una aplicaci�n Java, que funciona adem�s, como HAL\footnote{\textit{Hardware Abstraction Layer}, significa: capa de abstracci�n de hardware.}.
\item 
El \textit{port} de la capa SCJ de la m�quina virtual de HVM, para desarrollar aplicaciones Java SCJ.
\item 
La integraci�n manual del \textit{port} para la CIAA al IDE de HVM y la descripci�n de los pasos necesarios para llevar a cabo un proyecto con HVM.
\end{itemize}

\noindent Para validar el desarrollo se presentan:

\begin{itemize}
\item
Ejemplos de aplicaciones Java utilizando perif�ricos de la CIAA-NXP y EDU-CIAA-NXP mediante la biblioteca desarrollada.
\item
Un ejemplo de aplicaci�n Java SCJ utilizando el concepto de Proceso SCJ para demostrar el funcionamiento del cambio de contexto.
\item
Otro ejemplo de aplicaci�n Java SCJ que usa un Planificador SCJ.
\item 
Una aplicaci�n SCJ completa.
\end{itemize}

%% Faltar�an poner los resultados , que creo que ser�a la validaci�n

En conclusi�n, se obtiene de este Trabajo Final un entorno de desarrollo para aplicaciones Java SCJ sobre las plataformas CIAA-NXP y EDU-CIAA-NXP, que adem�s de ser software libre, cubre las necesidades planteadas, tanto al ofrecer programaci�n orientada a objetos, as� como funcionalidades de tiempo real para entornos industriales, sobre sistemas embebidos.

%\end{abstract}