\subsection{M�quinas Virtuales Java para aplicaciones de tiempo real} \label{sec:VMsJava}

Para poder utilizar Java sobre una plataforma de hardware en particular, se debe contar con una implementaci�n de la JVM conforme a alguna de las especificaciones anteriores. Se exponen las distintas m�quinas virtuales de Java que se consideraron y sus caracter�sticas:

\begin{itemize}
\item 
\textbf{JamaicaVM}. Sitio Web: \url{http://www.aicas.com/jamaica.html}.
\item 
\textbf{FijiVM}. Sitio Web: \url{http://fiji-systems.com/}.
\item 
\textbf{oSCJ}. Sitio Web: \url{https://www.cs.purdue.edu/sss/projects/oscj/}.
\item 
\textbf{KESO VM}. Sitio Web: \url{https://www4.cs.fau.de/Research/KESO/}.
\item 
\textbf{HVM}. Sitio Web: \url{http://icelab.dk/}.
\end{itemize}

\textbf{JamaicaVM} soporta la especificaci�n RTSJ. Es un desarrollo de la empresa Aicas, planeada para aplicaciones \textit{Hard Real-Time}, que posee un \textit{garbage collector} determin�stico (\textit{fully preemptable}). Se encuentra en estado de certificaci�n para su utilizaci�n en autom�viles y aviones. Si bien es la JVM m�s prometedora, la misma es de c�digo privado y por eso se descarta su utilizaci�n en este trabajo.

\medskip

\textbf{FijiVM} soporta la especificaci�n SCJ con muy buenas pretaciones, sin embargo al igual que JamaicaVM es un desarrollo de c�digo privado. 

% http://rtjava.blogspot.com.ar/2011/07/fijivm-real-time-java-vm-overview.html

\medskip 

\textit{Open Safety-Critical Java Implementation} (\textbf{oSCJ}) es un desarrollo de la universidad de Purdue, de c�digo abierto, que implementa un conjunto restringido de la especificaci�n SCJ, con foco en el nivel 0 de la misma. Posee un desarrollo de \textit{Technology Compatibility Kit} (TCK) como es solicitado en SCJ, chequeo est�tico de \textit{Annotations} SCJ y un conjunto de \textit{benchmarks} SCJ. Su licencia es \textit{New BSD}. La plataforma sobre la cual est� desarrollada oSCJ es una FPGA Xilinx con un softcore LEON3 corriendo el sistema operativo de tiempo real RTEMS. Este desarrollo dista mucho del microcontrolador que se utiliza en el trabajo y no se ha encontrado documentaci�n para portarlo a otra arquitectura.

\medskip 

\textbf{KESO VM} se desarrolla en Universidad Friedrich-Alexander, Alemania, con licencia LGPL V3. Est� dise�ada para correr sobre el sistema operativo de tiempo real OSEK, sobre las plataformas JOSEK, CiAO, Trampoline OS, Elektrobit ProOSEK y RTA-OSEK. En la web ofical existen ejemplos sobre la arquitectura AVR de 8 bits de la compa�ia Atmel. Si bien posee soporte de \textit{threads real-time}, no se basa en ninguna de las especificaciones para Java de tiempo real anteriores. Por otro lado, no se encontr� documentaci�n acerca de como llevar la misma a otra distribuci�n de OSEK.

\medskip 

Hardware near Virtual Machine (\textbf{HVM}) comenz� como un desarrollo para la tesis doctoral de Stephan Erbs Korsholm (Icelabs). Es un entorno de ejecuci�n de \textit{Safety Critical} Java (SCJ) nivel 1 y 2, de c�digo abierto dise�ado para plataformas embebidas de bajos recursos.

\medskip

HVM corre directamente sobre el hardware sin necesidad de un sistema operativo (\textit{bare-metall}). Su dise�o y exelente documentaci�n (v�ase [\label{bib:HVMref}]) facilita la portabilidad a nuevas arquitecturas.

\medskip

Se compone de las siguientes partes:
\begin{itemize}
\item \textbf{Icecaptools}. Es un \textit{plugin} que convierte al IDE Eclipse, en un IDE para la programaci�n en lenguaje Java para HVM. Icecaptools genera c�digo C a partir de la aplicaci�n Java de usuario para correr sobre la m�quina virtual de Java, HVM, as� como los proprios archivos que implementan a esta m�quina virtual.
\item \textbf{HVM SDK}. Es el \textit{Software Development Kit} de HVM que incluye las clases que implementan SCJ.
\end{itemize}

En [\ref{bib:PaperSCJKorsholm}] se porveen \textit{benckmarks} de HVM, KESO VM y FijiVM relativos a la aplicaci�n de los mismos en lenguaje C.

\medskip

Debido a estas caracter�sticas se elije HVM como la JVM a utilizar en este trabajo. 

\medskip

Cabe destacar, que durante el desarrollo del presente trabajo se ha entrado en contacto con Korsholm, v�a correo electr�nico, quien con excelente predisposici�n ha facilitado mucho la labor contestando todas las dudas. De esta manera, se ha logrado una cooperaci�n entre los equipos de investigaci�n de la Universidad Nacional de Quilmes e Icelabs, y se espera luego de la conclusi�n de este trabajo, contribuir al proyecto HVM con el aporte del \textit{port} para la CIAA de HVM.
