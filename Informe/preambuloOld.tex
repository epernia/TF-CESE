\documentclass[11pt,a4paper]{book}
\usepackage[spanish]{babel} % lo inclu� 20-08-2013
\usepackage[latin1]{inputenc}
%\usepackage[utf8]{inputenc}
\usepackage[a4paper]{geometry}
\geometry{verbose,tmargin=3cm,bmargin=2cm,lmargin=2.25cm,rmargin=2.25cm}

\usepackage[usenames,dvipsnames,table]{xcolor}

\usepackage{amsmath}
\usepackage{amsfonts}
\usepackage{amssymb}
\usepackage{enumerate}
\usepackage{graphicx}
%\usepackage{times}
%\usepackage{mathptmx}

% Para alterar los caption de las figuras
\usepackage[margin=2cm,font=footnotesize,labelfont=bf]{caption} 

% Encabezado y pie de p�gina personalizado
%\usepackage{fancyhdr} 

% Para sacar los encabezados y pies de p�gina en p�ginas en blanco
\usepackage{emptypage}

% Paquete para c�digo en alg�n lenguaje de programaci�n
\usepackage{listings} % Include the listings-package, lo inclu� 20-08-2013
\lstdefinestyle{customc}{
	frame=Ltb,
    framerule=0pt,
    aboveskip=0.5cm,
    framextopmargin=3pt,
    framexbottommargin=3pt,
    framexleftmargin=0.4cm,
    framesep=0pt,
    rulesep=.4pt,
    %backgroundcolor=\color{gray97},
    rulesepcolor=\color{black},
    %
    %stringstyle=\ttfamily,
    showstringspaces = false,
    basicstyle=\small\ttfamily,
    %commentstyle=\color{gray45},
    %keywordstyle=\bfseries,
    %
    numbers=left,
    numbersep=15pt,
    numberstyle=\tiny,
    numberfirstline = false,
    breaklines=true,
    %
    %     
  belowcaptionskip=1\baselineskip,
  %breaklines=true,
  %frame=L,
  xleftmargin=\parindent,
  language=C,
  %showstringspaces=false,
  %basicstyle=\footnotesize\ttfamily,
  keywordstyle=\bfseries\color{purple!80!black},
  commentstyle=\itshape\color{green!40!black},
  identifierstyle=\color{blue},
  stringstyle=\color{orange},
  %numbers=left, % where to put the line-numbers; values are (none, left, right)
  %numbersep=10pt,   % how far the line-numbers are from the code
  %numberstyle=\tiny % the style that is used for the line-numbers
}	
\lstset{escapechar=@,style=customc}	


% Paquete para links dentro del pdf
\usepackage{hyperref}
%\hypersetup{colorlinks=true}
\hypersetup{
    colorlinks,
    citecolor=cyan,
    filecolor=orange,
    linkcolor=red,	%black
    urlcolor=blue
}


% Renombro Comandos, es por si no est� BABEL
\renewcommand{\contentsname}{\'INDICE}
\renewcommand{\partname}{Parte}
\renewcommand{\appendixname}{Ap\'endice}
\renewcommand{\figurename}{Figura}
\renewcommand{\tablename}{Tabla}
%\renewcommand{\abstractname}{RESUMEN}
%\renewcommand{\refname}{REFERENCIA BIBLIOGR\'AFICA}
\renewcommand{\chaptername}{Cap\'{\i}tulo}



% Encabezado y pie de p�gina personalizado
%\pagestyle{fancy}
%\lhead{Trabajo Final}
%\chead{}
%\rhead{\textbf{\thepage}}
%\lfoot{Eric Nicol�s Pernia}
%\cfoot{}
%\rfoot{UNQ}
%\renewcommand{\footrulewidth}{0.04pt}
%\fancyhfoffset{0.2cm} %ancho del header/foot


% Caratula 1 - Caratula con \title
%\title{
%	Trabajo Final - Ingenier�a en Automatizaci�n y Control Industrial
%	\begin{center}
%	\includegraphics[width=55mm,height=60mm]{figuras/unq.png}
%	\end{center}
%	\author{Apellido: Pernia\\Nombre: Eric\\N�mero de legajo: 15925}
%	\date{\empty}
%}




% CONTADORES DE PAGINAS Y SUB SUB SUB SECCIONES

\setcounter{secnumdepth}{4}
\setcounter{tocdepth}{4}

\makeatletter

\newcounter {subsubsubsection}[subsubsection]

\renewcommand\thesubsubsubsection{\thesubsubsection .\@alph\c@subsubsubsection}

\newcommand
\subsubsubsection{\@startsection{subsubsubsection}{4}{\z@}%
	{-3.25ex\@plus -1ex \@minus -.2ex}%
	{1.5ex \@plus .2ex}%
	{\normalfont\normalsize\bfseries}}

\renewcommand\paragraph{\@startsection{paragraph}{5}{\z@}%
	{3.25ex \@plus1ex \@minus.2ex}%
	{-1em}%
	{\normalfont\normalsize\bfseries}}

\renewcommand\subparagraph{\@startsection{subparagraph}{6}{\parindent}%
	{3.25ex \@plus1ex \@minus .2ex}%
	{-1em}%
	{\normalfont\normalsize\bfseries}}

\newcommand*\l@subsubsubsection{\@dottedtocline{4}{10.0em}{4.1em}}

\renewcommand*\l@paragraph{\@dottedtocline{5}{10em}{5em}}
\renewcommand*\l@subparagraph{\@dottedtocline{6}{12em}{6em}}
\newcommand*{\subsubsubsectionmark}[1]{}
\makeatother



% MACROS
\input{MacrosDeEric}
%\eric{.} Mi macro, pone el texto azul
\input{MacrosDeCarlos}