\chapter{IMPLEMENTACI�N} \label{cap:implementacion}
% C�mo realic� concretamente el dise�o

%La implementaci�n consiste en:
%
%\begin{itemize}
%\item 
%La realizaci�n del \textit{port} de la m�quina virtual de HVM para que corra sobre el microcontrolador NXP LPC4337, que contienen las plataformas CIAA-NXP y EDU-CIAA-NXP, permitiendo la programaci�n de aplicaciones Java.
%\item 
%Una biblioteca sencilla para permitir controlar el Hardware desde una aplicaci�n Java, que funciona adem�s, como HAL\footnote{\textit{Hardware Abstraction Layer}, significa: capa de abstracci�n de hardware.}.
%\item 
%El \textit{port} de la capa SCJ de la m�quina virtual de HVM, para permitir desarrollar aplicaciones Java SCJ.
%\item 
%La integraci�n del \textit{port} para la CIAA al IDE de HVM, para completar un IDE de Java SCJ sobre la CIAA.
%\end{itemize}
%
%En las siguientes secciones se describen cada una de sus partes.

\section{Arquitectura del \textit{port} de HVM para las plastaformas CIAA} \label{sec:implArqHVM}

Para la implementaci�n de este trabajo se utiliza una arquitectura en capas con responsabilidades e interfaces definidas. El esquema general se presenta en la figura  [\ref{fig:ArquitecturaPort10x9}]. 

\begin{figure}[!htbp]
\begin{center}
\includegraphics*[width=10cm,height=9cm]{figuras/ArquitecturaPort10x9.png}
\par\caption{Arquitectura del port de HVM para las plastaformas CIAA.}\label{fig:ArquitecturaPort10x9}
\end{center}
\end{figure}

Las capas preexistentes son \textbf{icecap SDK}, \textbf{HVM}, \textbf{HVM SCJ} que corresponden a las distintas partes de que conforman HVM; y \textbf{LPCOpen}, la biblioteca provista por NXP para el microcontrolador LPC4337. Las capas desarrolladas se describen a continuaci�n.

\bigskip
\titulo{Vector de interrupciones}

Esta capa se compone de un �nico archivo \(IsrVector.c\) que contiene el vector de interrupciones. Para definir una nueva interrupci�n se debe agregar el nombre de la funci�n correspondiente al \textit{handler} de la misma en el vector, y luego definir dicha funci�n. El �nico \textit{handler} utilizado es el de interrupci�n de \textbf{SysTick}.

\newpage
\titulo{Biblioteca de Perif�ricos C}

La capa Biblioteca de Perif�ricos C, implementa la biblioteca de perif�ricos a bajo nivel, est� dise�ada para abstraer el hardware y presentar una API sencilla para el usuario. Se detalla en el apartado [\ref{sec:implBiblioPerif}]. Utiliza la biblioteca LPCOpen y el vector de interrupci�n.

\bigskip
\titulo{Capas que implementan la HAL de HVM para LPC4337}

La HAL de HVM a bajo nivel se implementa en dos capas, una que solo puede acceder a la Biblioteca de Perif�ricos C, compuesta por los archivos,

\begin{itemize}
\item
\textbf{LPC4337\_peripheral\_natives.c}. Esta capa simplemente implementa las funciones que respetan la firma de los m�todos nativos y llaman a sus respectivas funciones de la Biblioteca de Perif�ricos C.
\item
\textbf{LPC4337\_natives.c}. Contiene las funciones necesarias para el funcionamiento b�sico de HVM, se detalla en la secci�n [\ref{sec:HVMenLPC4337}].
\item
\textbf{LPC4337\_natives\_SCJ.c}. Resuelve las funciones de temporizaci�n para el \textbf{Planificador SCJ}. Utiliza el m�dulo \textit{Tick.c} de la capa Biblioteca de Perif�ricos C. En la secci�n [\ref{sec:HVMscjEnLPC4337}] se comunica su implementaci�n. 
\end{itemize}

y otra capa que accede directamente al hardware, implementada en \textit{assembler} (archivo LPC4337 \_interrupt.s). Est� �ltima que resuelve el cambio de \textit{threads} guardando el contexto y es la base del funcionamiento del m�dulo \textbf{Proceso SCJ}. 

\bigskip
\titulo{Biblioteca de Perif�ricos Java}

Corresponde a la implementaci�n de la biblioteca de perif�ricos en Java que finalmente dispondr� el usuario de HVM. Sus m�todos nativos se encuentran implementados en \textit{LPC4337\_peripheral\_natives.c}.

\section{Port b�sico de HVM al microcontrolador NXP LPC4337} \label{sec:HVMenLPC4337}

Para realizar el \textit{port} de HVM se deben efectuar una serie de tareas las cuales se han descripto en la secci�n [\ref{sec:disenoPortHvm}]. Se informa a continuaci�n c�mo ha sido llevado a cabo el subconjunto de las mismas, necesarias para la primer etapa del \textit{port}, que consiste en poder ejecutar programas Java (no SCJ) en la CIAA.

%--------------------------------------------------
\bigskip
\titulo{Obtenci�n de un entorno de desarrollo de C para la plataforma}

Para esta tarea se utilizaron las herramientas libres, provistas por el Proyecto CIAA, para el desarrollo en C sobre sus plataformas de hardware. Existe un instalador sencillo para Windows \textbf{CIAA-IDE-Suite 1.2.2} disponible para su descarga en: \url{https://github.com/ciaa/Software-IDE/releases/tag/v1.2.2}. Este paquete de software incluye:

\begin{itemize}
\item
\textbf{Consola cygwin} (con paquetes: PERL, PHP, gcc x86, arm-none-ebi-gcc y ruby). Esta consola permite compilar un programa en C, generando un ejecutable para el microcontrolador LPC4337. Tambi�n habilita la utilizaci�n de openOCD.
\item
\textbf{openOCD 0.9 x86/x64}. Se emplea tanto para al descarga del ejecutable al microcontrolador, como para depuraci�n mediante comandos GDB.
\item
\textbf{FTDI drivers libusb 1.0}. Se debe instalar por �nica vez el \textit{driver} seg�n la versi�n del sistema operativo. Reemplaza el \textit{driver} que Windows instala por defecto para el chip FTDI que se utilizan para \textit{debugger} en ambas plataformas.
\item
\textbf{Eclipse (Luna)}. IDE para desarrollo de programas en lenguaje C para la CIAA. Desde el mismo se pueden gestionar proyectos, y permite la compilaci�n y \textit{debug} en un entono gr�fico (utiliza cygwin por debajo).
\item
\textbf{Firmware 0.6.1}. Firmware para utilizar las plataformas CIAA.
\item
\textbf{IDE4PLC v1.0.2}. Software-PLC. No utilizado en este trabajo final.
\item
\textbf{Uninstaller}. Desinstalador del paquete.
\end{itemize}

Para probar su correcto funcionamiento, se debe crear un proyecto de CIAA Firmware. Pueden realizarse dos tipos de proyectos, los que incluyen \textbf{OSEK} (el RTOS\footnote{Siglas de Sistema Operativo de Tiempo Real.} de la CIAA, presentado en la secci�n [\ref{sec:disenoAPI}]) y los proyectos \textbf{\textit{bare-metal}} (programas que corren directo sobre el hardware sin la gesti�n de un Sistema Operativo). Como no se necesita OSEK, se realiz� un proyecto \textit{bare-metal}, este consiste en la utilizaci�n de una estructura definida de carpetas donde existe una carpeta principal que contiene el nombre del proyecto y sub-carpetas:

\begin{itemize}
%\item
\item
\textbf{inc}. Aqu� deben colocarse los \textit{headers} del proyecto (archivos \textit{.h}).
\item
\textbf{src}. En esta carpeta se alojan los archivos fuentes en del proyecto (de extensi�n \textit{.c}).
\item
\textbf{mak}. Contiene el \textit{makefile} del proyecto (archivo \textit{Makefile}). En el mismo debe indicarse los m�dulos de Firmware de la CIAA que se desean agregar al proyecto as� como los archivos a incluir en la compilaci�n.
\end{itemize}

Para realizarlo, se parte del proyecto plantilla llamado \textit{bare-metal}, que se incluye entre los ejemplos provistos (ubicados en \textit{Firmware/examples}). Esta plantilla contiene un archivo que define la funci�n \textit{main}, e incluye a la biblioteca del microcontrolador \textit{LPCOpen}, y otros archivos donde se define la tabla de vectores de interrupci�n y algunos \textit{handlers} por defecto de las mismas.

\medskip

Se crea un peque�o programa que hace destellar un \textit{led} (\textit{Blinky}) de la plataforma EDU-CIAA-NXP. Luego se modifica el archivo \textit{makefile.mine} (ubicado en la raiz de \textit{Firmware}) con los par�metros de plataforma a utilizar (\textit{BOARD}) y la ruta dle proyecto creado (\textit{PROJECT\_PATH}).
Finalmente, mediante \textbf{cygwin}, se compila y descarga a la plataforma EDU-CIAA-NXP. 

\medskip

\noindent \textbf{NOTA:} Se cuenta con la plataforma EDU-CIAA-NXP desde el comienzo del trabajo. En el caso de la CIAA-NXP se obtuvo casi a la finalizaci�n del mismo. Sin embargo, esto s�lo influye en la definici�n de la biblioteca de perif�ricos, para todas las dem�s tareas, el uso de cualquiera de las dos plataformas es indistinto.

%--------------------------------------------------
\bigskip
\titulo{Construcci�n de un comando de compilaci�n para compilar y \textit{linkear} los archivos generados por HVM}

En el apartado [\ref{sec:disenoHVMutilizacion}] se introduce como realizar un proyecto b�sico con HVM. Partiendo del comando de compilaci�n presentado, se observa cuales son los archivos que compila, as� como las definiciones necesarias. 

\medskip

Se crea un nuevo proyecto \textit{bare-metal} y se ingresan los datos necesarios al \textit{makefile} obteni�ndose: 

\begin{lstlisting}[language=Bash, numbers=left]
# Project Name: based on Project Path and used to define OSEK configuration file
PROJECT_NAME  =  $(lastword $(subst $(DS), , $(PROJECT_PATH)))

# Project path
# Defined $(PROJECT_PATH) in makefile.mine

#HVM files

vpath classes.c ..
vpath icecapvm.c ..
vpath methodinterpreter.c ..
vpath methods.c ..
vpath icecapvm.c ..
vpath gc.c ..
vpath natives_allOS.c ..
vpath allocation_point.c ..
vpath print.c ..
vpath $(PROJECT_PATH)$(DS)src

HVM_PATH = $(PROJECT_PATH)$(DS)hvm

HVM_SRC_FILES = $(HVM_PATH)$(DS)classes.c \
$(HVM_PATH)$(DS)icecapvm.c \
$(HVM_PATH)$(DS)methodinterpreter.c \
$(HVM_PATH)$(DS)methods.c \
$(HVM_PATH)$(DS)gc.c \
$(HVM_PATH)$(DS)natives_allOS.c \
$(HVM_PATH)$(DS)allocation_point.c \
$(HVM_PATH)$(DS)print.c

CFLAGS  +=  -DJAVA_HEAP_SIZE=$(JAVA_HEAP_SIZE) \
-DJAVA_STACK_SIZE=$(JAVA_STACK_SIZE) $(ADDITIONAL_FLAGS)

# source path
$(PROJECT_NAME)_SRC_PATH  += $(PROJECT_PATH)$(DS)src

# include path
INC_FILES  +=  $(PROJECT_PATH)$(DS)inc

# library source files
SRC_FILES  +=  $(wildcard $($(PROJECT_NAME)_SRC_PATH)$(DS)*.c) \
$(HVM_SRC_FILES)

# Modules needed for this example
MODS ?= externals$(DS)drivers   \
        modules$(DS)sapi      
\end{lstlisting}

% vpath LPC4337_interrupt.s ..
% $(PROJECT_PATH)$(DS)src$(DS)LPC4337_interrupt.s

N�tese que los archivos generados por HVM deben estar en una carpeta llamada \textbf{hvm} dentro del proyecto.

%--------------------------------------------------
\bigskip
\titulo{Definici�n de la plataforma en el archivo \textbf{ostypes.h}}

Para la definici�n de la plataforma se investig� la arquitectura del microcontrolador LPC4337 y su compilador \textit{arm-none-eabi-gcc} para definir los distintos tipos de datos, resultando en:

\begin{lstlisting}[language=C, numbers=left]
#if defined(LPC4337_TYPES_FOR_HVM)
   #undef PACKED
   #define PACKED
   #define DATAMEM __attribute__ ((section (".data")))
   #define PROGMEM
   #define RANGE
   #define pgm_read_byte(x) *((unsigned char*)x)
   #define pgm_read_word(x) *((unsigned short*)x)
   #define pgm_read_pointer(x, typeofx) *((typeofx)x)
   #define pgm_read_dword(x) pgm_read_pointer(x, uint32*)
   typedef int int32;
   typedef unsigned int uint32;
   typedef unsigned int pointer;
#endif
\end{lstlisting}

%--------------------------------------------------
\bigskip
\titulo{Definici�n de las funciones espec�ficas de la plataforma. Archivo \textbf{LPC4337\_natives.c}}

Como ha sido adelantado, este archivo implementa la parte de la HAL de HVM para que funcionen los programas b�sicos de Java sobre esta m�quina virtual.

\begin{itemize}
\item
\textbf{void init\_compiler\_specifics(void)}. Esta funci�n inicializa la plataforma CIAA. Primero se inicializa el \textbf{\textit{clock}} del sistema. Luego la \textbf{UART 2}. En ambas plataformas CIAA, la UART 2 del microcontolador LPC4337 se encuentra conectada al chip FTDI del \textit{debugger} y puede accederse v�a USB en el mismo conector que se usa para depuraci�n de la plataforma. De esta manera, es ideal para su utilizaci�n como salida de consola de HVM.
\item
\textbf{Pila de Java}. En este archivo se define la pila de Java como un vector cuyo tama�o ser� definido en el \textit{makefile.mine}, esto es: \textbf{int32 java\_stack[JAVA\_STACK\_SIZE];}
\item
\textbf{int32* get\_java\_stack\_base(int16 size)}. Devuelve un puntero a la pila de Java definida por el vector anterior: \textbf{return (int32*) java\_stack;}.
\item
\textbf{void initNatives(void)}. En esta funci�n se configura el resto de los perif�ricos de la biblioteca de C incluyendo el \textbf{SysTick} para interrumpir cada 10ms. Este perif�rico lo utilizan las funciones SCJ para manejar el tiempo del sistema.
\item
\textbf{void mark\_error(void)}, \textbf{void mark\_success(void)}. Como no se emplea el sistema de pruebas de regresi�n en esta primera iteraci�n, se dejan vac�as.
\item
\textbf{read/writeXXToIO}. No se utilizan objetos hardware para tratar con los registros debido a la implementaci�n de la biblioteca de C para perif�ricos. En consecuencia no se requieren.
\item
\textbf{init\_memory\_lock, lock\_memory, unlock\_memory}. Para las pruebas iniciales se dejan vac�as pues el �nico \textit{handler} de interrupci�n que hay es el de \textit{SysTick} que no asigna memoria.
\item
\textbf{void sendbyte(unsigned char byte)}. Imprime el \textit{byte} recibido como par�metro a trav�s de la UART 2.
\end{itemize}

%--------------------------------------------------
\bigskip
\titulo{Comprobaci�n inicial de funcionamiento}

Una vez completado el \textit{port} b�sico se realiza una prueba inicial del sistema. Para llevarla a cabo, se utiliza el programa \textit{``Hola Mundo HVM''} dado en [\ref{sec:disenoHVMutilizacion}]. Luego se crea un nuevo proyecto y dentro una carpeta nombrada \textbf{hvm}, se copian los archivos generados por HVM. El archivo \textbf{LPC4337\_natives.c} se copia a la carpeta \textbf{src} del proyecto. En la figura [\ref{fig:pathsProjectoJavaFirmware}] se muestra la estructura de carpetas resultante.

\begin{figure}[!htbp]
\begin{center}
\includegraphics*[width=16.5cm,height=6.5cm]{figuras/pathsProjectoJavaFirmware16,5x6,5.png}
\par\caption{Estructura de proyecto de firmware CIAA ``Hola Mundo con HVM''.}\label{fig:pathsProjectoJavaFirmware}
\end{center}
\end{figure}

Para su compilaci�n se debe modificar el archivo \textit{makefile.mine}.
Adem�s de definir \textit{BOARD} y \textit{PROJECT\_PATH} se debe agregar,

\begin{itemize}
\item
La cantidad de memoria para el \textit{Heap} de Java: \textbf{JAVA\_HEAP\_SIZE ?= 16384}.
\item
El tama�o de la Pila de Java: \textbf{JAVA\_STACK\_SIZE ?= 2048}.
\item
Algunos \textit{Flags} adicionales, siendo el m�s importante \textbf{-DLPC4337\_TYPES\_FOR\_HVM} que representa la arquitectura para la cual se va a compilar HVM.
\end{itemize}

obteni�ndose:

\begin{lstlisting}[language=Bash, numbers=left]
BOARD	?=	edu_ciaa_nxp
PROJECT_PATH ?= ..$(DS)HVM_Projects$(DS)holaMundoHVM

JAVA_HEAP_SIZE ?= 16384
JAVA_STACK_SIZE ?= 2048
ADDITIONAL_FLAGS ?= -g -Os -DLPC4337_TYPES_FOR_HVM -I..
\end{lstlisting}

Se compila con \textbf{cygwin} y se descarga a la plataforma. Despu�s se conecta a una Terminal Serie, se ingresan los par�metros de conexi�n y finalmente se resetea la EDU-CIAA-NXP placa para recibir el mensaje ``Hola Mundo HVM'' en la Terminal.


\section{Implementaci�n de la biblioteca para manejo de perif�ricos} \label{sec:implBiblioPerif}

\section{Biblioteca para manejo de perif�ricos (C)} \label{sec:implBiblioC}

%Implementaci�n de API para manejo de perif�ricos CIAA-NXP.

%sAPI

%Como subproducto, se obtiene adem�s un nuevo m�dulo de Firmware para la CIAA nombrado sAPI (simple API) que permite mediante un conjunto de clases encapsular el manejo de perif�ricos con una interfaz muy sencilla pensada para el programador con perfil inform�tico no experto en Sistemas embebidos. 

M�dulo \textbf{sAPI}:
Est� compuesta por los m�dulos:

\begin{verbatim}
sAPI.h
sAPI_Board.c, sAPI_Board.h
sAPI_DataTypes.h
sAPI_Delay.c, sAPI_Delay.h
sAPI_DigitalIO.c, sAPI_DigitalIO.h
sAPI_IsrVector.c, sAPI_IsrVector.h
sAPI_Tick.c, sAPI_Tick.h
\end{verbatim}

Este actualiza una variable global y luego ejecuta una funci�n cuya direcci�n est� contenida en una variable global para permitir engancharse a la interrupci�n de tick.


\section{Biblioteca para manejo de perif�ricos (Java)} \label{sec:implBiblioJava}

Est� formada por los m�dulos Device.Java, Pin.Java, Peripheral.Java, DigitalIO.Java, AnalogIO.Java, Uart.Java, Delay.Java, Button.Java y Led.Java. 



	\subsection{CIAA-NXP}
         %Mapeo de pines
         
	\subsection{EDU-CIAA}
         %Mapeo de pines
         
	\subsection{Java} %  ---> Como se ve la API desde Java Space


\section{Port de HVM SCJ al microcontrolador NXPLPC4337} \label{sec:HVMscjEnLPC4337}

%--------------------------------------------------
\bigskip
\titulo{Definir las funciones espec�ficas de la plataforma}

\textbf{LPC4337\_natives\_SCJ.c}


\textbf{LPC4337\_interrupt.s}

	\subsection{Funciones para SCJ}


%\section{Integraci�n y uso de las herramientas desarrolladas} \label{sec:ImplIde}


