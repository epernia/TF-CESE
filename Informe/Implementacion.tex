\chapter{IMPLEMENTACI�N} \label{cap:implementacion}
% C�mo realic� concretamente el dise�o

La implementaci�n consiste en:

\begin{itemize}
\item 
La realizaci�n del \textit{port} de la m�quina virtual de HVM para que corra sobre el microcontrolador NXP LPC4337, que contienen las plataformas CIAA-NXP y EDU-CIAA-NXP, permitiendo la programaci�n de aplicaciones Java.
\item 
Una API\footnote{\textit{Application Programming Interface}, es decir, una interfaz de programaci�n de aplicaciones.} sencilla para permitir controlar el Hardware desde una aplicaci�n Java, que funciona adem�s, como HAL\footnote{\textit{Hardware Abstraction Layer}, significa: capa de abstracci�n de hardware.}.
\item 
El \textit{port} de la capa SCJ de la m�quina virtual de HVM, para permitir desarrollar aplicaciones Java SCJ.
\item 
La integraci�n del \textit{port} para la CIAA al IDE de HVM, para completar un IDE de Java SCJ sobre la CIAA.
\end{itemize}

En las siguientes secciones se describen las herramientas utilizadas e implementaciones concretas del dise�o propuesto.

%\input{Lenguajes}

\section{Herramientas utilizadas} \label{sec:heramientas}

	\subsection{Lenguaje C}
	
	\subsection{Lenguaje Java}
	
	\subsection{Cygwin}
	
	\subsection{Eclipse}
   
\section{Port de HVM al microcontrolador NXP LPC4337} \label{sec:HVMenLPC4337}

\section{Implementaci�n de una API para manejo de perif�ricos} \label{sec:lenguajesJavaYC}

	\subsection{CIAA-NXP}
         %Mapeo de pines
         
	\subsection{EDU-CIAA}
         %Mapeo de pines
         
	\subsection{Java} %  ---> Como se ve la API desde Java Space

\section{Port de HVM SCJ al microcontrolador NXPLPC4337} \label{sec:HVMscjEnLPC4337}

	\subsection{Funciones para SCJ}
   
\section{Implementaci�n del IDE} \label{sec:ImplIde}

	\subsection{Plugins desarrollados}


%Implementaci�n de API para manejo de perif�ricos CIAA-NXP.

%sAPI

%Como subproducto, se obtiene adem�s un nuevo m�dulo de Firmware para la CIAA nombrado sAPI (simple API) que permite mediante un conjunto de clases encapsular el manejo de perif�ricos con una interfaz muy sencilla pensada para el programador con perfil inform�tico no experto en Sistemas embebidos. 

%	Enumerar el hardware que permite atacar
