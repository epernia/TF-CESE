%para gerentes}
\chapter{INTRODUCCI�N GENERAL}

\section{Marco tem�tico\: Programaci�n Orientada a Objetos en Sistemas embebidos para aplicaciones industriales}

Marco tem�tico: Programaci�n Orientada a Objetos en Sistemas embebidos para aplicaciones industriales.

\section{Plataforma CIAA}

Plataforma CIAA-NXP.

%http://proyecto-ciaa.com.ar/devwiki/doku.php?id=desarrollo:hardware:ciaa_nxp:ciaa_nxp_inicio

Plataforma EDU-CIAA-NXP.

%http://proyecto-ciaa.com.ar/devwiki/doku.php?id=desarrollo:edu-ciaa:edu-ciaa-nxp

\section{Lenguaje Java}

Se elije Java debido a que es uno de lenguajes de POO m�s utilizados en la actualidad. Es un lenguaje moderno, con manejo de memoria autom�tico y sintaxis similar a C++. A diferencia de otros, en lugar de compilarse para la arquitectura del controlador objetivo, se compila a un lenguaje similar al assembler (bytecodes) que se ejecuta mediante una M�quina Virtual, o VM, por sus siglas en ingl�s (Virtual Machine). Esta m�quina virtual corre habitualmente sobre un sistema operativo.




\section{M�quinas Virtuales de Java para sistemas embebidos}

M�quinas Virtuales de Java para sistemas embebidos.

\section{Especificaci�n SCJ}

Especificaci�n SCJ.

\section{Justificaci�n}

Con la creciente complejidad de las aplicaciones a realizar sobre sistemas embebidos en entornos industriales, y el aumento de las capacidades de memoria y procesamiento de los mismos, se desea poder aplicar t�cnicas avanzadas de programaci�n para realizar programas f�cilmente mantenibles, extensibles y reconfigurables. El paradigma de Programaci�n Orientada a Objetos (POO) cumple con estos requisitos.

Para aplicaciones industriales es requerimiento fundamental cumplir con especificaciones de tiempo real. Es por esto que se debe elegir un lenguaje POO que soporte de manejo de threads real-time. Ejemplos de aplicaciones con estos requerimientos son, control a lazo cerrado, etc.

\section{Objetivos}

Los objetivos del presente trabajo final son:

\begin{itemize}
\item 
Realizar el \textit{porting} de HVM para que corra sobre el microcontrolador NXP LPC4337 que contienen las plataformas CIAA-NXP y EDU-CIAA-NXP para permitir correr aplicaciones Java sobre la CIAA.
\item 
Dise�ar e implementar una HAL para controlar el Hardware desde una aplicaci�n Java mediante una API establecida.
\item 
Realizar el \textit{porting} de la capa SCJ de HVM para que corra sobre el microcontrolador NXP LPC4337 permitiendo aplicaciones Java SCJ sobre la CIAA.
\item 
Integrar en Icecaptools el \textit{porting} de HVM para completar el IDE de Java SCJ sobre la CIAA.
\end{itemize}

%\section{Alcance}

