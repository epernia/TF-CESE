%para gerentes}
\chapter{INTRODUCCI�N GENERAL - (8 pag)}

\section{Marco tem�tico\: Programaci�n Orientada a Objetos en Sistemas embebidos para aplicaciones industriales}

Con la creciente complejidad de las aplicaciones a realizar sobre sistemas embebidos en entornos industriales, y el aumento de las capacidades de memoria y procesamiento de los mismos, se desea poder aplicar t�cnicas avanzadas de programaci�n para realizar programas f�cilmente mantenibles, extensibles y reconfigurables. El paradigma de Programaci�n Orientada a Objetos (POO) cumple con estos requisitos.

\medskip

Para aplicaciones industriales es requerimiento fundamental cumplir con especificaciones de tiempo real. Es por esto que se debe elegir un lenguaje POO que soporte de manejo de threads real-time. 
Un ejemplo de aplicaciones con estos requerimientos son las de control a lazo cerrado. En estas aplicaciones es necesario garantizar una taza de muestreo peri�dica uniforme para poder aplicar la teor�a de control autom�tico.

\medskip

En las siguientes secciones se expondr�n las distintas tecnolog�as que constituyen esta tem�tica.

\subsection{Lenguajes de POO para sistemas embebidos}

En la actualidad existen varios desarrollos de lenguajes de programaci�n orientado a objetos para sistemas embebidos, entre ellos podemos citar:

\begin{itemize}
\item 
C++.
\item 
Java.
\item 
MicroPhyton.
\end{itemize}

\subsection{Plataforma CIAA}

\paragraph{CIAA-NXP}

\medskip

Plataforma CIAA-NXP.

%http://proyecto-ciaa.com.ar/devwiki/doku.php?id=desarrollo:hardware:ciaa_nxp:ciaa_nxp_inicio

\paragraph{EDU-CIAA-NXP}

\medskip

Plataforma EDU-CIAA-NXP.

%http://proyecto-ciaa.com.ar/devwiki/doku.php?id=desarrollo:edu-ciaa:edu-ciaa-nxp


\section{Justificaci�n}

Ac� deber�a poner por que se elije Java y HVM. Arrancamos con Java, que se elije porque es superador a C++ y existe la especificaci�n RTSJ.

Luego se descubre HVM que implementa SCJ.

\subsection{Lenguaje de POO Java}

El lenguje Java fue desarrollado originalmente por James Gosling de Sun Microsystems y publicado en 1995 como un componente fundamental de la plataforma Java de Sun Microsystems (actualmente fue adquirida por la compa��a Oracle). 

\medskip

\noindent Las caracter�sticas principales del mismo son:

\begin{itemize}
\item 
Lenguaje de programaci�n de prop�sito general.
\item 
Orientado a objetos.
\item 
Independiente de la m�quina.
\item 
Seguro para trabajar en red.
\item 
Potente para sustituir c�digo nativo.
\item 
Con comprobaci�n estricta de tipos.
\item 
Manejo de memoria autom�tico mediante Recolector de Basura.
\item 
Sin punteros, utiliza �nicamente referencias a objetos.
\item 
Permite programaci�n concurrente de forma est�ndar.
\end{itemize}

\medskip

Para lograr la independencia de la m�quina posee la caracter�stica de ser un lenguaje compilado e interpretado. Todo programa en Java, se compila primero a un lenguaje similar a un \textit{assembler} gen�rico basando en pila (\textit{bytecodes}), que luego es interpretado por una m�quina virtual de Java (JVM) dependiente de la plataforma. 

\medskip

La JVM es habitualmente un programa que corre sobre un sistema operativo, sin embargo, existen implementaciones de la JVM que corren directamente sobre el hardware (\textit{bare-metal}) y procesadores capaces de ejecutar \textit{bytecodes} de Java directamente, por ejemplo, el microcontrolador ARM926EJ-S.

\medskip

Debido a estas caracter�sticas y la existencia de la especificaci�n RTSJ que contempla aplicaciones \textit{Real-Time} (se desarrolla en la secci�n [\ref{sec:EspSCJ}]), se ha elegido Java como lenguaje POO para la programaci�n de la CIAA. Adem�s, Java es uno de lenguajes de POO m�s utilizados en la actualidad a nivel mundial por programadores inform�ticos.


% PONER COMO BIBLIO DE JAVA - EL LIBRO DE JUAREZ
% PONER COMO BIBLIO DE JAVA - http://www3.uji.es/~belfern/pdidoc/IX26/Documentos/introJava.pdf
% Procesador Java - http://www.atmel.com/images/arm_926ejs_trm.pdf




\subsection{Especificaciones RTSJ y SCJ}\label{sec:EspSCJ}

Especificaci�n SCJ.


Executive Summary

This Safety-Critical Java Specification (JSR-302), based on the Real-Time Specification for Java (JSR-1), defines a set of Java services that are designed to be usable by applications requiring some level of safety certification. The specification is targeted to a wide variety of very demanding certification paradigms such as the safety-critical requirements of DO-178B, Level A.

This specification presents a set of Java classes providing for safety-critical application startup, concurrency, scheduling, synchronization, input/output, memory management, timer management, interrupt processing, native interfaces, and exceptions.
To enhance the certifiability of applications constructed to conform to this specification, this specification also presents a set of annotations that can be used to permit static checking for applications to guarantee that the application exhibits certain safety properties.

To enhance the portability of safety-critical applications across different implemen- tations of this specification, this specification also lists a minimal set of Java libraries that must be provided by conforming implementations.


\section{Elecci�n de la VM de Java, HVM} \label{sec:eleccionHvm}

HVM\footnote{Significa que se ejecuta directamente sobre el hardware, sin necesidad de un sistema operativo.} son las siglas de Hardware Virtual Machine. Es una m�quina virtual (VM) de Java libre, dise�ada para ejecutarse bare metal sobre microcontroladores y portable a diferentes arquitecturas. Su implementaci�n de referencia fue realizada por Stephan Erbs Korsholm (Icelab, Dinamarca) para su tesis doctoral. Esta VM cumple con la especificaci�n Safety-Critical Java (SCJ) Level 0, 1 y 2 permitiendo realizar aplicaciones en tiempo real para sistemas cr�ticos. Provee tres formas de acceso al hardware:

\begin{itemize}
\item 
Variables Bindeadas.
\item 
Hardware Objects.
\item 
M�todos nativos.
\end{itemize}

La �ltima forma, m�todos nativos, se elige como alternativa para proveer al programa de usuario en lenguaje Java el acceso al hardware.

\medskip

Por que HVM y no otra VM de Java.

\section{Objetivos}

Los objetivos del presente trabajo final son:

\begin{itemize}

\item 
Realizar el \textit{port} de la m�quina virtual de HVM para que corra sobre las plataformas CIAA-NXP y EDU-CIAA-NXP, permitiendo la programaci�n de aplicaciones Java.
\item 
Dise�ar e implementar una API sencilla para permitir controlar perif�ricos del micronctrolador desde una aplicaci�n Java.
\item 
Llevar a cabo el \textit{port} de la capa SCJ de la m�quina virtual de HVM, para permitir desarrollar aplicaciones Java SCJ.
\item 
La integraci�n del \textit{port} para la CIAA al IDE de HVM, para completar un IDE de Java SCJ sobre la CIAA.
\end{itemize}

%\section{Alcance}
